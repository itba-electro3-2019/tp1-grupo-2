\noindent
A partir del an\'alisis de la tabla \ref{tab:truth_table} fue posible expresar el valor de cada bit de salida en función de los mintérminos de los bits de entrada. Estas expresiones se encuentran formuladas a continuaci\'on:\\

\noindent
\small
$Y_0 (A,B,C,D) = A.\overline{B}.\overline{C}.\overline{D}
 +  A.\overline{B}.\overline{C}.D
 +  A.\overline{B}.C.\overline{D}
 +  A.\overline{B}.C.D
 +  A.B.\overline{C}.\overline{D}
 +  A.B.\overline{C}.D
 +  A.B.C.\overline{D}
 +  A.B.C.D $
\vspace{8mm}\par

\noindent
\small
$Y_1 (A,B,C,D) = \overline{A}.B.\overline{C}.\overline{D}
 + \overline{A}.B.\overline{C}.D
 + \overline{A}.B.C.\overline{D}
 + \overline{A}.B.C.D
 + A.\overline{B}.\overline{C}.\overline{D}
 + A.\overline{B}.\overline{C}.D
 + A.\overline{B}.C.\overline{D}
 + A.\overline{B}.C.D $
\vspace{8mm}\par

\noindent
\small
$Y_2 (A,B,C,D) = \overline{A}.\overline{B}.C.\overline{D}
 + \overline{A}.\overline{B}.C.D
 + \overline{A}.B.\overline{C}.\overline{D}
 + \overline{A}.B.\overline{C}.D
 + A.\overline{B}.C.\overline{D}
 + A.\overline{B}.C.D
 + A.B.\overline{C}.\overline{D}
 + A.B.\overline{C}.D $
\vspace{8mm}\par


\noindent
\small
$Y_3 (A,B,C,D) = \overline{A}.\overline{B}.\overline{C}.D
 + \overline{A}.\overline{B}.C.\overline{D}
 + \overline{A}.B.\overline{C}.D
 + \overline{A}.B.C.\overline{D}
 + A.\overline{B}.\overline{C}.D
 + A.\overline{B}.C.\overline{D}
 + A.B.\overline{C}.D
 + A.B.C.\overline{D} $
\vspace{8mm}\par

