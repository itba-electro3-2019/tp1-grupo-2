\subsubsection{Resoluci\'on mediante algebra booleana}
\vspace{5mm}
\noindent
La expresi\'on obtenida por los maxt\'erminos dados es la siguiente:\\


$f(d,c,b,a)=(d+c+b+a)\cdot(d+c+\overline{b}+a)\cdot(d+\overline{c}+b+a)\cdot(d+\overline{c}+\overline{b}+\overline{a})\cdot(\overline{d}+c+b+a)\cdot(\overline{d}+c+\overline{b}+a)\cdot(\overline{d}+\overline{c}+b+a) $
\vspace{5mm}


\noindent
De aqu\'i utilizando la propiedad de absorci\'on podremos reducir la expresi\'on a:\par
\vspace{5mm}

$f(d,c,b,a)=\color{red}{(d+c+b+a)}\color{black}\cdot\color{red}{(d+c+\overline{b}+a)}\color{black}\cdot\color{blue}{(d+\overline{c}+b+a)}\color{black}\cdot\color{black}{(d+\overline{c}+\overline{b}+\overline{a})}\color{black}\cdot\color{green}{(\overline{d}+c+b+a)}\color{black}\cdot\color{green}{(\overline{d}+c+\overline{b}+a)}\color{black}\cdot\color{blue}{(\overline{d}+\overline{c}+b+a)}$
\vspace{8mm}\par
$f(d,c,b,a)=\color{red}{(d+c+a)}\color{black}\cdot\color{blue}{(\overline{c}+b+a)}\color{black}\cdot\color{black}{(d+\overline{c}+\overline{b}+\overline{a})}\color{black}\cdot\color{green}{(\overline{d}+c+a)}$
\vspace{8mm}\par

\noindent
De esta forma, aprovechando el hecho de que x.x=x lo que haremos ser\'a traer los t\'erminos: $(d+c+b+a) y (\overline{d}+c+b+a)$ de la primer linea de donde partimos como productos, de esta forma obtendremos la siguiente expresi\'on:\par
\vspace{5mm}

$f(d,c,b,a)={(d+c+a)}\cdot{(\overline{c}+b+a)}\cdot{(d+\overline{c}+\overline{b}+\overline{a})}\cdot{(\overline{d}+c+a)}\cdot{(d+c+b+a)}\cdot{(\overline{d}+c+b+a)}$
\vspace{8mm}\par

\noindent
Ahora utilizando nuevamiente la propiedad de absorci\'on tendremos:\par
\vspace{5mm}

$f(d,c,b,a)=\color{red}{(d+c+a)}\color{black}\cdot{(\overline{c}+b+a)}\cdot{(d+\overline{c}+\overline{b}+\overline{a})}\cdot\color{red}{(\overline{d}+c+a)}\color{black}\cdot\color{blue}{(d+c+b+a)}\color{black}\cdot\color{blue}{(\overline{d}+c+b+a)}$
\vspace{8mm}\par

$f(d,c,b,a)=\color{red}{(c+a)}\color{black}\cdot{(\overline{c}+b+a)}\cdot{(d+\overline{c}+\overline{b}+\overline{a})}\cdot\color{blue}{(c+b+a)}$
\vspace{8mm}\par

\noindent
\color{black}Nuevamente realizamos una \'ultima absorci\'on:\par\vspace{5mm}

$f(d,c,b,a)=\color{black}{(c+a)}\color{black}\cdot\color{red}{(\overline{c}+b+a)}\color{black}\cdot{(d+\overline{c}+\overline{b}+\overline{a})}\cdot\color{red}{(c+b+a)}$
\vspace{8mm}\par


$f(d,c,b,a)=\color{black}{(c+a)}\color{black}\cdot\color{red}{(b+a)}\color{black}\cdot{(d+\overline{c}+\overline{b}+\overline{a})}$
\vspace{8mm}\par


\noindent
\color{black}Finalmente, la expresi\'on hallada es:\par\vspace{5mm}

$f(d,c,b,a)=\color{black}{(c+a)}\cdot{(b+a)}\cdot{(d+\overline{c}+\overline{b}+\overline{a})}$
\vspace{8mm}\par