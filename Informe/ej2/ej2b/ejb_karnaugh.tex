\subsubsection{Resoluci\'on mediante mapas de Karnaugh}


%Empty Karnaugh map 4x4
\newenvironment{Karnaugh2}%
{
\begin{tikzpicture}[baseline=(current bounding box.north),scale=0.8]
\draw (0,0) grid (4,4);
\draw (0,4) -- node [pos=0.7,above right,anchor=south west] {dc} node [pos=0.7,below left,anchor=north east] {ba} ++(135:1);
%
\matrix (mapa) [matrix of nodes,
        column sep={0.8cm,between origins},
        row sep={0.8cm,between origins},
        every node/.style={minimum size=0.3mm},
        anchor=8.center,
        ampersand replacement=\&] at (0.5,0.5)
{
                       \& |(c00)| 00         \& |(c01)| 01         \& |(c11)| 11         \& |(c10)| 10         \& |(cf)| \phantom{00} \\
|(r00)| 00             \& |(0)|  \phantom{0} \& |(1)|  \phantom{0} \& |(3)|  \phantom{0} \& |(2)|  \phantom{0} \&                     \\
|(r01)| 01             \& |(4)|  \phantom{0} \& |(5)|  \phantom{0} \& |(7)|  \phantom{0} \& |(6)|  \phantom{0} \&                     \\
|(r11)| 11             \& |(12)| \phantom{0} \& |(13)| \phantom{0} \& |(15)| \phantom{0} \& |(14)| \phantom{0} \&                     \\
|(r10)| 10             \& |(8)|  \phantom{0} \& |(9)|  \phantom{0} \& |(11)| \phantom{0} \& |(10)| \phantom{0} \&                     \\
|(rf) | \phantom{00}   \&                    \&                    \&                    \&                    \&                     \\
};
}%
{
\end{tikzpicture}
}



\begin{center}
\begin{Karnaugh2}
    \minterms{4,5,6,7,9,11,12,14,15}
    \maxterms{0,1,8,3,2,13,10}
    \implicant{0}{2}{red}
    \implicantcantons[2pt]{blue}
    \implicantsol{13}{green}
\end{Karnaugh2}
\end{center}
\par
\noindent
De aqu\'i, mediante la separaci\'on por grupos marcada previamente en los mapas de Karnaugh, se observa de cuales variables depende cada grupo y si estas est\'an negadas o no, llegando a la siguiente expresi\'on: \par\vspace{5mm}

$f(d,c,b,a)=\color{black}{(c+a)}\cdot{(b+a)}\cdot{(d+\overline{c}+\overline{b}+\overline{a})}$
\vspace{8mm}\par
\noindent
Como era de esperarse, el resultado obtenido mediante mapas de Karnaugh es igual a la simplificaci\'on obtenida mediante algebra booleana.