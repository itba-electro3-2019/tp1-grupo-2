%%%%%%%%%%%%%%%%%%%%%%%%%%%%%%%%%%%%%%%%%
% University Assignment Title Page 
% LaTeX Template
% Version 1.0 (27/12/12)
%
% This template has been downloaded from:
% http://www.LaTeXTemplates.com
%
% Original author:
% WikiBooks (http://en.wikibooks.org/wiki/LaTeX/Title_Creation)
%
% License:
% CC BY-NC-SA 3.0 (http://creativecommons.org/licenses/by-nc-sa/3.0/)
% 
% Instructions for using this template:
% This title page is capable of being compiled as is. This is not useful for 
% including it in another document. To do this, you have two options: 
%
% 1) Copy/paste everything between \begin{document} and \end{document} 
% starting at \begin{titlepage} and paste this into another LaTeX file where you 
% want your title page.
% OR
% 2) Remove everything outside the \begin{titlepage} and \end{titlepage} and 
% move this file to the same directory as the LaTeX file you wish to add it to. 
% Then add \input{./title_page_1.tex} to your LaTeX file where you want your
% title page.
%
%%%%%%%%%%%%%%%%%%%%%%%%%%%%%%%%%%%%%%%%%
%\title{Title page with logo}
%----------------------------------------------------------------------------------------
%	PACKAGES AND OTHER DOCUMENT CONFIGURATIONS
%----------------------------------------------------------------------------------------

\documentclass[12pt, a4paper]{article}





\usepackage{graphicx}
\usepackage[spanish]{babel}
\usepackage[utf8x]{inputenc}
\usepackage{amsmath}
\usepackage[table,xcdraw]{xcolor}

\usepackage[colorinlistoftodos]{todonotes}
\usepackage{siunitx}
\usepackage{lipsum}
\usepackage{setspace}
\usepackage{caption}
\usepackage{physics}
\usepackage{tikz}
\usetikzlibrary{matrix,calc}
\captionsetup[table]{position=bottom}   %% or below
\makeatletter
\newcommand{\mypm}{\mathbin{\mathpalette\@mypm\relax}}
\newcommand{\@mypm}[2]{\ooalign{%
  \raisebox{.1\height}{$#1+$}\cr
  \smash{\raisebox{-.6\height}{$#1-$}}\cr}}
\makeatother
\usepackage{lscape}
\newcommand\tab[1][1cm]{\hspace*{#1}}

\begin{document}

\begin{titlepage}

\newcommand{\HRule}{\rule{\linewidth}{0.5mm}} % Defines a new command for the horizontal lines, change thickness here

\center % Center everything on the page
 
%----------------------------------------------------------------------------------------
%	HEADING SECTIONS
%----------------------------------------------------------------------------------------

\textsc{\LARGE  Instituto Tecnológico de Buenos Aires}\\[1.5cm] % Name of your university/college
\textsc{\large Electrónica III}\\[0.5cm] % Minor heading such as course title

%----------------------------------------------------------------------------------------
%	TITLE SECTION
%----------------------------------------------------------------------------------------

\HRule \\[0.4cm]
{ \huge \bfseries Trabajo Práctico N°1}\\[0.4cm] % Title of your document
\HRule \\[1.5cm]
 \textsc{\large Grupo 2}\\[0.5cm] 
%----------------------------------------------------------------------------------------
%	AUTHOR SECTION
%----------------------------------------------------------------------------------------

\begin{minipage}{0.5\textwidth}
\begin{flushleft} \large
\emph{Integrantes:}\\
Santiago \textsc{Arribere, 59169} \newline % Your name
Gonzalo \textsc{Davidov, 59117}\newline
R. Nicolás \textsc{Trozzo, 59434}\newline
Matías \textsc{Francois, 59828}\newline
Pablo \textsc{Scheinfeld, 59065}

\end{flushleft}
\end{minipage}
~
\begin{minipage}{0.4\textwidth}
\begin{flushright} \large
\emph{Profesores:} \\
Kevin \textsc{Dewald}\\ % Supervisor's Name
Pablo Enrique \textsc{Wundes}\\
Miguel Pablo \textsc{Aguirre}
\end{flushright}
\end{minipage}\\[1cm]

% If you don't want a supervisor, uncomment the two lines below and remove the section above
%\Large \emph{Author:}\\
%John \textsc{Smith}\\[3cm] % Your name

%----------------------------------------------------------------------------------------
%	DATE SECTION
%----------------------------------------------------------------------------------------

{\large 4 de septiembre de 2019}\\[1cm] % Date, change the \today to a set date if you want to be precise

%----------------------------------------------------------------------------------------
%	LOGO SECTION
%----------------------------------------------------------------------------------------

%\includegraphics[scale=0.2]{images/caratula/logo2.png}\\[1cm] % Include a department/university logo - this will require the graphicx package
 
%----------------------------------------------------------------------------------------

\vfill % Fill the rest of the page with whitespace

\end{titlepage}

\tableofcontents
\newpage
\input{introduccion.tex}
\section{Ejercicio 1}


\section{Ejercicio 2}
\noindent
El ejercicio 2 consta en resolver dos problemas, el primero, un ejercicio de 5 variables, dado en funci\'on de mint\'erminos y el segundo, de 4 variables, en funci\'on de maxt\'erminos.\par
\subsection{Primera parte}

\begin{displaymath}
f(A,B,C,D,E) = \sum{m(0,2,4,7,8,10,12,16,18,20,23,24,25,26,27,28)}\par
\end{displaymath}
\subsubsection{Resoluci\'on mediante algebra booleana}
\vspace{5mm}
\noindent
La expresi\'on obtenida por los minterminos dados es la siguiente:\\

$f(A,B,C,D,E) =  \overline{A}.\overline{B}.\overline{C}.\overline{D}.\overline{E} + \overline{A}.\overline{B}.\overline{C}.D.\overline{E} +
\vspace{5mm}\overline{A}.\overline{B}.C.\overline{D}.\overline{E} + \overline{A}.\overline{B}.C.D.E + \overline{A}.B.\overline{C}.\overline{D}.\overline{E} + \overline{A}.B.\overline{C}.D.\overline{E} + \vspace{5mm}\overline{A}.B.C.\overline{D}.\overline{E} + A.\overline{B}.\overline{C}.\overline{D}.\overline{E} + A.\overline{B}.\overline{C}.D.\overline{E} + A.\overline{B}.C.\overline{D}.\overline{E} + A.\overline{B}.C.D.E + A.B.\overline{C}.\overline{D}.\overline{E} + A.B.\overline{C}.\overline{D}.E + A.B.\overline{C}.D.\overline{E} + A.B.\overline{C}.D.E + A.B.C.\overline{D}.\overline{E} $
\vspace{5mm}

\noindent
Aplicando simplificaci\'ones del algebra booleana se puede reducir la expresi\'on:\\

$f(A,B,C,D,E) =  \overline{A}.\overline{B}.\overline{C}.\overline{D}.\overline{E} + \overline{A}.\overline{B}.\overline{C}.D.\overline{E} +
\vspace{5mm}\overline{A}.\overline{B}.C.\overline{D}.\overline{E} + \overline{A}.\overline{B}.C.D.E + \overline{A}.B.\overline{C}.\overline{D}.\overline{E} + \overline{A}.B.\overline{C}.D.\overline{E} + \vspace{5mm}\overline{A}.B.C.\overline{D}.\overline{E} + A.\overline{B}.\overline{C}.\overline{D}.\overline{E} + A.\overline{B}.\overline{C}.D.\overline{E} + A.\overline{B}.C.\overline{D}.\overline{E} + A.\overline{B}.C.D.E + A.B.\overline{C}.\overline{D}.\overline{E} + A.B.\overline{C}.\overline{D}.E + A.B.\overline{C}.D.\overline{E} + \vspace{5mm}A.B.\overline{C}.D.E + A.B.C.\overline{D}.\overline{E} + \textcolor{red}{A.\overline{B}.\overline{C}.\overline{D}.\overline{E} +
A.B.\overline{C}.\overline{D}.\overline{E} + \overline{A}.\overline{B}.\overline{C}.\overline{D}.\overline{E} + A.B.\overline{C}.D.\overline{E} + \overline{A}.B.\overline{C}.\overline{D}.\overline{E} + A.B.\overline{C}.\overline{D}.\overline{E}}$
\vspace{5mm}\par

\noindent
De aqu\'i utilizando la propiedad de absorci\'on \par
\vspace{5mm}

$f(A,B,C,D,E) =  \color{red}{\overline{A}.\overline{B}.\overline{C}.\overline{D}.\overline{E}} + \overline{A}.\overline{B}.\overline{C}.D.\overline{E} +
\vspace{5mm}\color{green}{\overline{A}.\overline{B}.C.\overline{D}.\overline{E}} + \color{blue}{\overline{A}.\overline{B}.C.D.E} + \color{yellow}{\overline{A}.B.\overline{C}.\overline{D}.\overline{E}} + \color{pink}{\overline{A}.B.\overline{C}.D.\overline{E}} + \vspace{5mm}\color{violet}{\overline{A}.B.C.\overline{D}.\overline{E}} + \color{brown}{A.\overline{B}.\overline{C}.\overline{D}.\overline{E}} + A.\overline{B}.\overline{C}.D.\overline{E} + \color{orange}{A.\overline{B}.C.\overline{D}.\overline{E}} + \color{blue}{A.\overline{B}.C.D.E} + \color{yellow}{A.B.\overline{C}.\overline{D}.\overline{E}} + \color{gray}{A.B.\overline{C}.\overline{D}.E} + \color{cyan}{A.B.\overline{C}.D.\overline{E}} + \color{gray}{A.B.\overline{C}.D.E} + \vspace{5mm}\color{orange}{A.B.C.\overline{D}.\overline{E}} + \color{magenta}{A.\overline{B}.\overline{C}.\overline{D}.\overline{E}} +
A.B.\overline{C}.\overline{D}.\overline{E} + \color{green}{\overline{A}.\overline{B}.\overline{C}.\overline{D}.\overline{E}} + \color{pink}{A.B.\overline{C}.D.\overline{E}} + \color{violet}{\overline{A}.B.\overline{C}.\overline{D}.\overline{E}} + \color{cyan}{A.B.\overline{C}.\overline{D}.\overline{E}}$
\vspace{8mm}\par
$
\color{black}{f(A,B,C,D,E) =}  \color{red}{\overline{A}.\overline{B}.\overline{C}.\overline{E}} +
\vspace{5mm}\color{green}{\overline{A}.\overline{B}.\overline{D}.\overline{E}} + \color{blue}{\overline{B}.C.D.E} + \color{yellow}{B.\overline{C}.\overline{D}.\overline{E}} + \color{pink}{B.\overline{C}.D.\overline{E}} + \color{violet}{\overline{A}.B.\overline{D}.\overline{E}} + \color{brown}{A.\overline{B}.\overline{C}.\overline{E}} + \color{orange}{A.C.\overline{D}.\overline{E}} + \color{gray}{A.B.\overline{C}.E} + \color{cyan}{A.B.\overline{C}.\overline{E}} + \color{magenta}{A.\overline{C}.\overline{D}.\overline{E}}$
\vspace{8mm}\par
$
\color{black}{f(A,B,C,D,E) =}  \overline{A}.\overline{B}.\overline{C}.\overline{E} +
\vspace{5mm}\overline{A}.\overline{B}.\overline{D}.\overline{E} + \overline{B}.C.D.E + B.\overline{C}.\overline{D}.\overline{E} + B.\overline{C}.D.\overline{E} + \overline{A}.B.\overline{D}.\overline{E} + A.\overline{B}.\overline{C}.\overline{E} + A.C.\overline{D}.\overline{E} + A.B.\overline{C}.E + A.B.\overline{C}.\overline{E} + A.\overline{C}.\overline{C}.\overline{E}$
\vspace{8mm}\par
$
\color{black}{f(A,B,C,D,E) =}  \color{red}{\overline{A}.\overline{B}.\overline{C}.\overline{E}} +
\vspace{5mm}\color{green}{\overline{A}.\overline{B}.\overline{D}.\overline{E}} + \color{black}{\overline{B}.C.D.E} + \color{blue}{B.\overline{C}.\overline{D}.\overline{E} + B.\overline{C}.D.\overline{E}} + \color{green}{\overline{A}.B.\overline{D}.\overline{E}} + \color{red}{A.\overline{B}.\overline{C}.\overline{E}} + \color{magenta}{A.C.\overline{D}.\overline{E}} + \color{cyan}{A.B.\overline{C}.E + A.B.\overline{C}.\overline{E}} + \color{magenta}{A.\overline{C}.\overline{D}.\overline{E}}$
\vspace{8mm}\par
$
\color{black}{f(A,B,C,D,E) =}  \color{red}{\overline{B}.\overline{C}.\overline{E}} + \color{green}{\overline{A}.\overline{D}.\overline{E}} + \color{black}{\overline{B}.C.D.E} + \color{blue}{B.\overline{C}.\overline{E}} + \color{magenta}{A.\overline{D}.\overline{E}} + \color{cyan}{A.B.\overline{C}}$
\vspace{8mm}\par
$
\color{black}{f(A,B,C,D,E) =}  \overline{B}.\overline{C}.\overline{E} + \overline{A}.\overline{D}.\overline{E} + \overline{B}.C.D.E + B.\overline{C}.\overline{E} + A.\overline{D}.\overline{E} + A.B.\overline{C}$
\vspace{8mm}\par
$
\color{black}{f(A,B,C,D,E) =}  \color{red}{\overline{B}.\overline{C}.\overline{E}} + \color{blue}{\overline{A}.\overline{D}.\overline{E}} + \color{black}{\overline{B}.C.D.E} + \color{red}{B.\overline{C}.\overline{E}} + \color{blue}{A.\overline{D}.\overline{E}} + \color{black}{A.B.\overline{C}}$
\vspace{8mm}\par
\noindent
\color{black}Finalmente, la expresi\'on hallada es:\par\vspace{5mm}

$f(A,B,C,D,E) =  \overline{C}.\overline{E} + \overline{D}.\overline{E} + \overline{B}.C.D.E + A.B.\overline{C}$
\vspace{8mm}\par
\input{ej2/ej2a/ej2a_karnaugh.tex}
\subsubsection{Representaci\'on con compuertas AND, OR y NOT}
\noindent
La representaci\'on de la expresion obtenida previamente mediante compuertas logicas AND, OR y NOT se puede ver a continuaci\'on.
\newpage
\begin{figure}[h!]
    \centering
    \begin{minipage}{0.85\textwidth}
        \centering
        \includegraphics[width=0.9\textwidth]{images/ej2/ej2andornot .png} % first figure itself
         \label{fig:ej2andornot}
    \end{minipage}\hfill
\end{figure}

\subsubsection{Representaci\'on con compuertas NAND}
\noindent
La representaci\'on de la expresion obtenida previamente mediante compuertas logicas NAND se puede ver a continuaci\'on.

\begin{figure}[h!]
    \centering
    \begin{minipage}{0.85\textwidth}
        \centering
        \includegraphics[width=0.9\textwidth]{images/ej2/ej2nand.png} % first figure itself
         \label{fig:ej2nand}
    \end{minipage}\hfill
\end{figure}

\section{Ejercicio 3}
\noindent
El ejercicio consta en crear mediante el uso del lenguaje Verilog un encoder de 4 entradas y 2 salidas y un demux de 4 salidas.\par
Se tiene como consideración que ante cualquier entrada no esperada la salida será alta impedancia, es decir, Z.

\section{Ejercicio 4}
\subsection{Expresi\'on en funci\'on de los mint\'erminos}
\noindent
Con el fin de convertir un número binario de 4 bits en su equivalente en código de Gray, se plantearon todas las posibilidades en la siguiente tabla de verdad, donde $Y_0$, $Y_1$, $Y_2$ e $Y_3$ representan las correspondientes salidas y A, B, C y D las entradas del sistema. Ambas se encuentran mostradas en forma ordenada, tomando $Y_0$ como el bit m\'as significativo y $Y_3$ como el bit menos significativo. De la misma forma, para las entradas, A representa el bit m\'as significativo y D el menos significativo.

\begin{table}[h!]
\centering
\begin{tabular}{|cccc|cccc|}
\hline

\multicolumn{1}{|c}{A} & \multicolumn{1}{c}{B}   & \multicolumn{1}{c}{C}   & D                        & \multicolumn{1}{c}{$Y_0$}   & \multicolumn{1}{c}{$Y_1$}   & \multicolumn{1}{c}{$Y_2$}   & $Y_3$                        \\ \hline
\rowcolor[HTML]{34FF34} 
0                       & 0                        & 0                        & 0                        & \cellcolor[HTML]{FD6864}0 & \cellcolor[HTML]{FD6864}0 & \cellcolor[HTML]{FD6864}0 & \cellcolor[HTML]{FD6864}0 \\
\rowcolor[HTML]{34FF34} 
0                       & 0                        & 0                        & 1                        & \cellcolor[HTML]{FD6864}0 & \cellcolor[HTML]{FD6864}0 & \cellcolor[HTML]{FD6864}0 & \cellcolor[HTML]{FD6864}1 \\
\rowcolor[HTML]{34FF34} 
0                       & 0                        & 1                        & 0                        & \cellcolor[HTML]{FD6864}0 & \cellcolor[HTML]{FD6864}0 & \cellcolor[HTML]{FD6864}1 & \cellcolor[HTML]{FD6864}1 \\
\rowcolor[HTML]{34FF34} 
0                       & {\color[HTML]{333333} 0} & {\color[HTML]{333333} 1} & {\color[HTML]{333333} 1} & \cellcolor[HTML]{FD6864}0 & \cellcolor[HTML]{FD6864}0 & \cellcolor[HTML]{FD6864}1 & \cellcolor[HTML]{FD6864}0 \\
\rowcolor[HTML]{34FF34} 
0                       & {\color[HTML]{333333} 1} & {\color[HTML]{333333} 0} & {\color[HTML]{333333} 0} & \cellcolor[HTML]{FD6864}0 & \cellcolor[HTML]{FD6864}1 & \cellcolor[HTML]{FD6864}1 & \cellcolor[HTML]{FD6864}0 \\
\rowcolor[HTML]{34FF34} 
0                       & {\color[HTML]{333333} 1} & {\color[HTML]{333333} 0} & {\color[HTML]{333333} 1} & \cellcolor[HTML]{FD6864}0 & \cellcolor[HTML]{FD6864}1 & \cellcolor[HTML]{FD6864}1 & \cellcolor[HTML]{FD6864}1 \\
\rowcolor[HTML]{34FF34} 
0                       & {\color[HTML]{333333} 1} & {\color[HTML]{333333} 1} & {\color[HTML]{333333} 0} & \cellcolor[HTML]{FD6864}0 & \cellcolor[HTML]{FD6864}1 & \cellcolor[HTML]{FD6864}0 & \cellcolor[HTML]{FD6864}1 \\
\rowcolor[HTML]{34FF34} 
0                       & {\color[HTML]{333333} 1} & {\color[HTML]{333333} 1} & {\color[HTML]{333333} 1} & \cellcolor[HTML]{FD6864}0 & \cellcolor[HTML]{FD6864}1 & \cellcolor[HTML]{FD6864}0 & \cellcolor[HTML]{FD6864}0 \\
\rowcolor[HTML]{34FF34} 
1                       & {\color[HTML]{333333} 0} & {\color[HTML]{333333} 0} & {\color[HTML]{333333} 0} & \cellcolor[HTML]{FD6864}1 & \cellcolor[HTML]{FD6864}1 & \cellcolor[HTML]{FD6864}0 & \cellcolor[HTML]{FD6864}0 \\
\rowcolor[HTML]{34FF34} 
1                       & {\color[HTML]{333333} 0} & {\color[HTML]{333333} 0} & {\color[HTML]{333333} 1} & \cellcolor[HTML]{FD6864}1 & \cellcolor[HTML]{FD6864}1 & \cellcolor[HTML]{FD6864}0 & \cellcolor[HTML]{FD6864}1 \\
\rowcolor[HTML]{34FF34} 
1                       & {\color[HTML]{333333} 0} & {\color[HTML]{333333} 1} & {\color[HTML]{333333} 0} & \cellcolor[HTML]{FD6864}1 & \cellcolor[HTML]{FD6864}1 & \cellcolor[HTML]{FD6864}1 & \cellcolor[HTML]{FD6864}1 \\
\rowcolor[HTML]{34FF34} 
1                       & {\color[HTML]{333333} 0} & {\color[HTML]{333333} 1} & {\color[HTML]{333333} 1} & \cellcolor[HTML]{FD6864}1 & \cellcolor[HTML]{FD6864}1 & \cellcolor[HTML]{FD6864}1 & \cellcolor[HTML]{FD6864}0 \\
\rowcolor[HTML]{34FF34} 
1                       & {\color[HTML]{333333} 1} & {\color[HTML]{333333} 0} & {\color[HTML]{333333} 0} & \cellcolor[HTML]{FD6864}1 & \cellcolor[HTML]{FD6864}0 & \cellcolor[HTML]{FD6864}1 & \cellcolor[HTML]{FD6864}0 \\
\rowcolor[HTML]{34FF34} 
1                       & {\color[HTML]{333333} 1} & {\color[HTML]{333333} 0} & {\color[HTML]{333333} 1} & \cellcolor[HTML]{FD6864}1 & \cellcolor[HTML]{FD6864}0 & \cellcolor[HTML]{FD6864}1 & \cellcolor[HTML]{FD6864}1 \\
\rowcolor[HTML]{34FF34} 
1                       & {\color[HTML]{333333} 1} & {\color[HTML]{333333} 1} & {\color[HTML]{333333} 0} & \cellcolor[HTML]{FD6864}1 & \cellcolor[HTML]{FD6864}0 & \cellcolor[HTML]{FD6864}0 & \cellcolor[HTML]{FD6864}1 \\
\rowcolor[HTML]{34FF34} 
1                       & {\color[HTML]{333333} 1} & {\color[HTML]{333333} 1} & {\color[HTML]{333333} 1} & \cellcolor[HTML]{FD6864}1 & \cellcolor[HTML]{FD6864}0 & \cellcolor[HTML]{FD6864}0 & \cellcolor[HTML]{FD6864}0
\end{tabular}
\caption{\label{tab:truth_table}Tabla de verdad}
\end{table}
\noindent
A partir del an\'alisis de la tabla \ref{tab:truth_table} fue posible expresar el valor de cada bit de salida en función de los mintérminos de los bits de entrada. Estas expresiones se encuentran formuladas a continuaci\'on:\\

\noindent
\small
$Y_0 (A,B,C,D) = A.\overline{B}.\overline{C}.\overline{D}
 +  A.\overline{B}.\overline{C}.D
 +  A.\overline{B}.C.\overline{D}
 +  A.\overline{B}.C.D
 +  A.B.\overline{C}.\overline{D}
 +  A.B.\overline{C}.D
 +  A.B.C.\overline{D}
 +  A.B.C.D $
\vspace{8mm}\par

\noindent
\small
$Y_1 (A,B,C,D) = \overline{A}.B.\overline{C}.\overline{D}
 + \overline{A}.B.\overline{C}.D
 + \overline{A}.B.C.\overline{D}
 + \overline{A}.B.C.D
 + A.\overline{B}.\overline{C}.\overline{D}
 + A.\overline{B}.\overline{C}.D
 + A.\overline{B}.C.\overline{D}
 + A.\overline{B}.C.D $
\vspace{8mm}\par

\noindent
\small
$Y_2 (A,B,C,D) = \overline{A}.\overline{B}.C.\overline{D}
 + \overline{A}.\overline{B}.C.D
 + \overline{A}.B.\overline{C}.\overline{D}
 + \overline{A}.B.\overline{C}.D
 + A.\overline{B}.C.\overline{D}
 + A.\overline{B}.C.D
 + A.B.\overline{C}.\overline{D}
 + A.B.\overline{C}.D $
\vspace{8mm}\par


\noindent
\small
$Y_3 (A,B,C,D) = \overline{A}.\overline{B}.\overline{C}.D
 + \overline{A}.\overline{B}.C.\overline{D}
 + \overline{A}.B.\overline{C}.D
 + \overline{A}.B.C.\overline{D}
 + A.\overline{B}.\overline{C}.D
 + A.\overline{B}.C.\overline{D}
 + A.B.\overline{C}.D
 + A.B.C.\overline{D} $
\vspace{8mm}\par


\subsection{Desarrollo de los mapas de Karnaugh de las salidas}
\noindent
Para los siguientes mapas se utiliz\'o la misma convenci\'on que para la tabla de verdad. De esta forma el orden es A-B-C-D, siendo A el bit m\'as significativo y D el bit menos significativo.

%Empty Karnaugh map 4x4
\newenvironment{Karnaugh4}%
{
\begin{tikzpicture}[baseline=(current bounding box.north),scale=0.8]
\draw (0,0) grid (4,4);
\draw (0,4) -- node [pos=0.7,above right,anchor=south west] {AB} node [pos=0.7,below left,anchor=north east] {CD} ++(135:1);
%
\matrix (mapa) [matrix of nodes,
        column sep={0.8cm,between origins},
        row sep={0.8cm,between origins},
        every node/.style={minimum size=0.3mm},
        anchor=8.center,
        ampersand replacement=\&] at (0.5,0.5)
{
                       \& |(c00)| 00         \& |(c01)| 01         \& |(c11)| 11         \& |(c10)| 10         \& |(cf)| \phantom{00} \\
|(r00)| 00             \& |(0)|  \phantom{0} \& |(1)|  \phantom{0} \& |(3)|  \phantom{0} \& |(2)|  \phantom{0} \&                     \\
|(r01)| 01             \& |(4)|  \phantom{0} \& |(5)|  \phantom{0} \& |(7)|  \phantom{0} \& |(6)|  \phantom{0} \&                     \\
|(r11)| 11             \& |(12)| \phantom{0} \& |(13)| \phantom{0} \& |(15)| \phantom{0} \& |(14)| \phantom{0} \&                     \\
|(r10)| 10             \& |(8)|  \phantom{0} \& |(9)|  \phantom{0} \& |(11)| \phantom{0} \& |(10)| \phantom{0} \&                     \\
|(rf) | \phantom{00}   \&                    \&                    \&                    \&                    \&                     \\
};
}%
{
\end{tikzpicture}
}

\begin{center}
Mapa de Karnaugh para $Y_0$:

\begin{Karnaugh4}
    \centering
    \minterms{3,2,7,6,10,11,14,15}
    \maxterms{0,1,12,13,4,5,9,8}
    \implicant{3}{10}{red}
\end{Karnaugh4}
\end{center}

\noindent
Mediante la observaci\'on del mapa y teniendo en cuenta que para el conjunto marcado en rojo tanto B, como C y como D cambian de valor (entre 0 y 1) se pudo concluir que la expresi\'on m\'as simplificada para $Y_0$ es:
\begin{equation}
    Y_0 = A
    \label{ecy0}
\end{equation}

\begin{center}
Mapa de Karnaugh para $Y_1$:

\begin{Karnaugh4}
    \centering
    \minterms{1,2,5,6,13,14,9,10}
    \maxterms{0,11,3,4,7,8,12,15}
    \implicant{1}{9}{magenta}
    \implicant{2}{10}{yellow}
\end{Karnaugh4}
\end{center}

\noindent
En este caso, al realizar un an\'alisis del mapa se pudieron apreciar dos grupos (marcados en magenta y amarillo). Dentro del primer conjunto (magenta) sucede un cambio en los valores de C y D, manteni\'endose A en 0 y B en 1. Con respecto al segundo grupo (amarillo) C y D nuevamente toman distintos valores, mientras que A se mantiene en 1 y B en 0. Como conclusi\'on, la expresi\'on m\'as simplificada para $Y_1$ es: 
\begin{equation}
    Y_1 = \overline{A}.B
 + A.\overline{B}
 \label{ecy1}
\end{equation}

\begin{center}
Mapa de Karnaugh para $Y_2$:

\begin{Karnaugh4}
    \centering
    \minterms{1,3,5,7,12,14,8,10}
    \maxterms{0,2,4,6,9,11,13,15}
    \implicant{1}{7}{blue}
    \implicantcostats{12}{10}{gray}
\end{Karnaugh4}
\end{center}

\noindent
Con el fin de obtener la expresi\'on m\'as simplificada para $Y_2$ se distinguieron 2 grupos en el mapa. Para el caso del grupo azul, B se mantiene con el valor 1 y C con el valor 0 mientras que A y D cambian. Por otra parte, examinando el conjunto gris, se cumple que B vale 0 y C vale 1, siendo las \'unicas variables que no cambian su valor. De esta forma, la expresi\'on queda:
\begin{equation}
    Y_2 = B.\overline{C}
 + \overline{B}.C
 \label{ecy2}
\end{equation}
\begin{center}
Mapa de Karnaugh para $Y_3$:

\begin{Karnaugh4}
    \centering
    \minterms{4,5,7,6,8,9,11,10}
    \maxterms{0,2,1,3,12,14,13,15}
    \implicant{4}{6}{brown}
    \implicant{8}{10}{green}
\end{Karnaugh4}
\end{center}
\noindent

Por \'ultimo, tienendo en cuenta que: dentro del grupo marr\'on C vale 0 y D, 1; en el grupo verde, C es constantemente 1 y D, 0 -siendo las \'unicas variables que mantienen su valor en cada caso-, para $Y_3$ la expresi\'on m\'as simplificada es: 
 \begin{equation}
 Y_3 = \overline{C}.D
 + C.\overline{D}
     \label{ecy3}
 \end{equation}

\noindent
Por otra parte, analizando todas las expresiones se puede apreciar que, excluyendo la entrada A y la salida $Y_0$, el resto de las salidas son el resultado de ingresar 2 entradas a una compuerta XOR. Este hecho se puede pensar como dicha operaci\'on (XOR) es la que se realiza entre cada bit binario de la entrada y su bit inmediatamente anterior (m\'as significativo) para formar cada bit del n\'umero expresado por c\'odigo de Gray - exceptuando el primer bit ya que toman el mismo valor independientemente del resto-.

\subsection{Representaci\'on por medio de un circuito l\'ogico}

\noindent
A partir de las expresiones \ref{ecy0},  \ref{ecy1}, \ref{ecy2}, y \ref{ecy3}, se pudo plantear el comportamiento a trav\'es de un circuito l\'ogico. Para ello, se tomaron compuertas del tipo AND, OR y NOT. A continuaci\'on se expone dicho esquema.

\begin{figure}[h!]
    \centering
    \includegraphics[scale=0.6]{images/logiccircuitej4.png}
    \caption{Representaci\'on de las expresiones mediante un circuito l\'ogico}
    \label{fig:circuito4fig}
\end{figure}

\noindent
El circuito esquematizado en la figura \ref{fig:circuito4fig} es el resultado de expresiones del tipo de suma de productos. Este factor se puede observar en el diagrama debido a que, excepto en $Y_0$ su valor es directamente equivalente al de A, las salidas parten inmediatamente despu\'es de una compuerta OR cuyas entradas siempre son las salidas de dos compuertas AND. 
\subsection{Implementaci\'on en Verilog}
\noindent
En la presente secci\'on se explicita como fue el razonamiento l\'ogico llevado a cabo para la implementaci\'on del circuito en Verilog. Para realizar dicha tarea, se decidi\'o desarrollar un archivo con el m\'odulo principal del circuito y otro archivo con un banco de pruebas.

\noindent
Con respecto al archivo encargado del funcionamiento, lo primero que se realiz\'o fue observar las expresiones de las salidas del circuito para comprender de que manera relacionarlas con la entrada. Al ser 4 entradas y 4 salidas donde cada una representa un nibble, en Verilog se utilizaron arreglos de 4 bits para trabajar sobre ellas. Se define un arreglo como input y luego el otro como output. Por esta raz\'on, se debe tener en cuenta que de las salidas de la convenci\'on utilizada anteriormente, $Y_0$ ser\'a el bit 3 del arreglo de salida y $Y_3$ ser\'a el bit 0.

\noindent
A continuaci\'on, y teniendo en cuenta que $Y_0$ toma el valor de A y que ambos son el bit m\'as significativo de la salida y entrada respectivamente, se igual el bit 3 del arreglo de salida al bit 3 del arreglo de entrada.

\noindent
Como ya fue mencionado, el resto de las salidas pueden ser entendidas como el resultado de una compuerta XOR entre el bit de entrada en la misma posici\'on que el de salida y el de entrada en una posici\'on anterior (m\'as significativo). Este razonamiento se lleva a cabo con las 3 expresiones restantes, siendo estas muy simples al hacer uso de la operaci\'on $'\wedge'$ (XOR) de Verilog. As\'i, por ejemplo, el bit 1 de salida ($Y_2$) depende de un XOR entre el bit 1 y 2 de entrada (B y C).

\noindent
En relaci\'on al banco de pruebas, se utiliz\'o otro archivo .v donde a una entrada de 4 bits se analiza la salida resultante. Este proceso se realiza en el banco para todos los numeros binarios de 4 bits a la entrada y observando si la salida es su correspondiente c\'odigo de Gray. El resultado fue exitoso en todos los casos.

\section{Conclusi\'on}
\noindent
A modo de cierre, a lo largo del informe se pudieron desarrollar las tem\'aticas propuestas y cumplir los objetivos planteados.
Fue posible utilizar las herramientas fundamentales de la l\'ogica combinacional, es decir, tablas de verdad y mapas de Karnaugh,
para la implementaci\'on circuital gr\'afica o programada. En cuanto al lenguaje Verilog, se tuvo una primera aproximaci\'on
a los lenguajes de programaci\'on funcionales, mediante la implementaci\'on de m\'odulos. Adicionalmente, se desarrollaron bancos de 
pruebas para verificar el correcto funcionamiento de los m\'odulos. 
Por \'ultimo, para el trabajo fue utilizada la herramienta GitHub, muy utilizada en el \'ambito profesional para llevar un control
de versiones de los avances. 

\end{document}
