\section{Ejercicio 5}

\noindent
El \'ultimo ejercicio requiere la implementaci\'on en Verilog de un m\'odulo multiplicador de dos n\'umeros de un d\'igito en formato BCD, expresando a la salida el resultado como un n\'umero de dos d\'igitos en el mismo formato. \newline
El m\'odulo tiene 3 partes principales: la validaci\'on de la entrada, la multiplicaci\'on en s\'i, y la conversi\'on de binario a BCD del resultado de la multiplicaci\'on. A continuaci\'on se explican consideraciones generales del c\'odigo y las etapas de la implementaci\'on.

\subsection{Consideraciones generales}
\noindent
Se detallan las consideraciones que se tuvieron a la hora de programar para lograr una mayor calidad del c\'odigo. 
Hay un m\'odulo principal llamado 'BCDMultiplier', que recibe los n\'umeros BCD a multiplicar en dos arreglos de 4 bits, devuelve el resultado de la multiplicaci\'on en formato BCD en un arreglo de 8 bits, y adem\'as devuelve 2 bits m\'as de validaci\'on que son explicados mas adelante.
Todos los arreglos de bits fueron definidos siguiendo la conveci\'on Big Endian ([0:n]), en vez de Little Endian ([n:0]), para tener el bit m\'as significativo en la primera posici\'on, ya que se consider\'o m\'as intuitivo. \newline



\subsection{Validaci\'on de la entrada}
\noindent
Antes de proceder a realizar la multiplicaci\'on, es necesario validar que cada nybble recibido corresponda a un n\'umero BCD v\'alido, es decir, que corresponda  a un n\'umero menor a 10. Para poder determinar la validez de un n\'umero recibido, en la Tabla \ref{ej5table} se muestra la tabla de verdad de la validaci\'on, considerando al n\'umero como $X = x_1x_2x_3x_4$ donde se puede observar que la salida toma el valor 1, que se toma como error, si y s\'olo si el valor de la entrada supera el valor de 9, que es el l\'imite para la representaci\'on en un d\'igito BCD.

% Please add the following required packages to your document preamble:
% \usepackage[table,xcdraw]{xcolor}
% If you use beamer only pass "xcolor=table" option, i.e. \documentclass[xcolor=table]{beamer}
\begin{table}[ht]
\centering
\begin{tabular}{|l|c|c|c|c|c|}
\hline
N                        & $x_1$                        & $x_2$                        & $x_3$                        & $x_4$                        & y = $x_1x_2x_3x_4$ \textgreater $b1001$ \\ \hline
0                        & \cellcolor[HTML]{34FF34}0 & \cellcolor[HTML]{34FF34}0 & \cellcolor[HTML]{34FF34}0 & \cellcolor[HTML]{34FF34}0 & \cellcolor[HTML]{FD6864}0          \\ \hline
1                        & \cellcolor[HTML]{34FF34}0 & \cellcolor[HTML]{34FF34}0 & \cellcolor[HTML]{34FF34}0 & \cellcolor[HTML]{34FF34}1 & \cellcolor[HTML]{FD6864}0          \\ \hline
2                        & \cellcolor[HTML]{34FF34}0 & \cellcolor[HTML]{34FF34}0 & \cellcolor[HTML]{34FF34}1 & \cellcolor[HTML]{34FF34}0 & \cellcolor[HTML]{FD6864}0          \\ \hline
3                        & \cellcolor[HTML]{34FF34}0 & \cellcolor[HTML]{34FF34}0 & \cellcolor[HTML]{34FF34}1 & \cellcolor[HTML]{34FF34}1 & \cellcolor[HTML]{FD6864}0          \\ \hline
4                        & \cellcolor[HTML]{34FF34}0 & \cellcolor[HTML]{34FF34}1 & \cellcolor[HTML]{34FF34}0 & \cellcolor[HTML]{34FF34}0 & \cellcolor[HTML]{FD6864}0          \\ \hline
5                        & \cellcolor[HTML]{34FF34}0 & \cellcolor[HTML]{34FF34}1 & \cellcolor[HTML]{34FF34}0 & \cellcolor[HTML]{34FF34}1 & \cellcolor[HTML]{FD6864}0          \\ \hline
6                        & \cellcolor[HTML]{34FF34}0 & \cellcolor[HTML]{34FF34}1 & \cellcolor[HTML]{34FF34}1 & \cellcolor[HTML]{34FF34}0 & \cellcolor[HTML]{FD6864}0          \\ \hline
7                        & \cellcolor[HTML]{34FF34}0 & \cellcolor[HTML]{34FF34}1 & \cellcolor[HTML]{34FF34}1 & \cellcolor[HTML]{34FF34}1 & \cellcolor[HTML]{FD6864}0          \\ \hline
8                        & \cellcolor[HTML]{34FF34}1 & \cellcolor[HTML]{34FF34}0 & \cellcolor[HTML]{34FF34}0 & \cellcolor[HTML]{34FF34}0 & \cellcolor[HTML]{FD6864}0          \\ \hline
9                        & \cellcolor[HTML]{34FF34}1 & \cellcolor[HTML]{34FF34}0 & \cellcolor[HTML]{34FF34}0 & \cellcolor[HTML]{34FF34}1 & \cellcolor[HTML]{FD6864}0          \\ \hline
10                       & \cellcolor[HTML]{34FF34}1 & \cellcolor[HTML]{34FF34}0 & \cellcolor[HTML]{34FF34}1 & \cellcolor[HTML]{34FF34}0 & \cellcolor[HTML]{FD6864}1          \\ \hline
\multicolumn{1}{|c|}{11} & \cellcolor[HTML]{34FF34}1 & \cellcolor[HTML]{34FF34}0 & \cellcolor[HTML]{34FF34}1 & \cellcolor[HTML]{34FF34}1 & \cellcolor[HTML]{FD6864}1          \\ \hline
12                       & \cellcolor[HTML]{34FF34}1 & \cellcolor[HTML]{34FF34}1 & \cellcolor[HTML]{34FF34}0 & \cellcolor[HTML]{34FF34}0 & \cellcolor[HTML]{FD6864}1          \\ \hline
13                       & \cellcolor[HTML]{34FF34}1 & \cellcolor[HTML]{34FF34}1 & \cellcolor[HTML]{34FF34}0 & \cellcolor[HTML]{34FF34}1 & \cellcolor[HTML]{FD6864}1          \\ \hline
14                       & \cellcolor[HTML]{34FF34}1 & \cellcolor[HTML]{34FF34}1 & \cellcolor[HTML]{34FF34}1 & \cellcolor[HTML]{34FF34}0 & \cellcolor[HTML]{FD6864}1          \\ \hline
15                       & \cellcolor[HTML]{34FF34}1 & \cellcolor[HTML]{34FF34}1 & \cellcolor[HTML]{34FF34}1 & \cellcolor[HTML]{34FF34}1 & \cellcolor[HTML]{FD6864}1          \\ \hline

\end{tabular}
\label{ej5table}
\caption{Tabla de verdad de la validaci\'on de la entrada}
\end{table}

\noindent
De la tabla se obtiene el siguiente mapa de Karnaugh:


%Empty Karnaugh map 4x4
\newenvironment{Karnaugh3}%
{
\begin{tikzpicture}[baseline=(current bounding box.north),scale=0.8]
\draw (0,0) grid (4,4);
\draw (0,4) -- node [pos=0.7,above right,anchor=south west] {$x_3x_4$} node [pos=0.7,below left,anchor=north east] {$x_1x_2$} ++(120:1);
%
\matrix (mapa) [matrix of nodes,
        column sep={0.8cm,between origins},
        row sep={0.8cm,between origins},
        every node/.style={minimum size=0.3mm},
        anchor=8.center,
        ampersand replacement=\&] at (0.5,0.5)
{
                       \& |(c00)| 00         \& |(c01)| 01         \& |(c11)| 11         \& |(c10)| 10         \& |(cf)| \phantom{00} \\
|(r00)| 00             \& |(0)|  \phantom{0} \& |(1)|  \phantom{0} \& |(3)|  \phantom{0} \& |(2)|  \phantom{0} \&                     \\
|(r01)| 01             \& |(4)|  \phantom{0} \& |(5)|  \phantom{0} \& |(7)|  \phantom{0} \& |(6)|  \phantom{0} \&                     \\
|(r11)| 11             \& |(12)| \phantom{0} \& |(13)| \phantom{0} \& |(15)| \phantom{0} \& |(14)| \phantom{0} \&                     \\
|(r10)| 10             \& |(8)|  \phantom{0} \& |(9)|  \phantom{0} \& |(11)| \phantom{0} \& |(10)| \phantom{0} \&                     \\
|(rf) | \phantom{00}   \&                    \&                    \&                    \&                    \&                     \\
};
}%
{
\end{tikzpicture}
}

\begin{center}
\begin{Karnaugh3}
    \centering
    \maxterms{0,1,2,3,4,5,6,7,8,9}
    \minterms{10,11,12,13,14,15}
    \implicant{15}{10}{red}
    \implicant{12}{14}{blue}
\end{Karnaugh3}
\end{center}



\noindent
Resolviendo para los mint\'erminos, se llega a la siguiente suma de productos: 
\begin{equation}
    y = x_1\cdot x_2 + x_1\cdot x_3
    \label{ej5_validation_eq}
\end{equation}
\noindent
Por lo tanto, para validar los n\'umeros de entrada, basta implementar la Ecuaci\'on \ref{ej5_validation_eq}. \\
A la hora de la implementaci\'on en Verilog, se decidi\'o realizar un m\'odulo espec\'ifico que realice la validaci\'on, mediante la expresi\'on l\'ogica obtenida anteriormente. Esta validaci\'on se realiza para ambos n\'umeros de entrada, y el m\'odulo principal tiene dos bits de output para indicar la validez de cada par\'ametro de entrada.
\subsection{Multiplicaci\'on binaria y conversi\'on a BCD}
\noindent
Una vez validada la entrada, se debe realizar la multiplicaci\'on en binario. Para esto se hace uso del operador '*' de Verilog, que arroja el resultado de la multiplicaci\'on binaria. Teniendo el resultado en binario de la multiplicaci\'on, hay m\'ultiples formas de pasarlo a BCD. En primer lugar se hab\'ia decidido implementar el algoritmo DoubleDabble, pero la implementaci\'on no era acorde a la filosof\'ia de Verilog, ya que hab\'ia que realizar muchas iteraciones. Dado que realizar una tabla de verdad involucraba m\'as de 50 casos, se utiliz\'o el operador \% (m\'odulo) de Verilog, para obtener las decenas y las unidades.
\subsection{Banco de pruebas}
\noindent
El banco de pruebas prueba 6 casos que se consideran representativos de la totalidad de entradas que puede tener el m\'odulo. En primer lugar, se prueba si una multiplicaci\'on por 0 arroja 0 como resultado. Los siguientes dos casos son multiplicaciones v\'alidas con resultado distinto de 0. Por \'ultimo, hay 3 casos inv\'alidos, en los que se puede observar que el m\'odulo distingue cu\'al de sus entradas es inv\'alida. \\
Para que el c\'odigo del banco de pruebas no fuera repetitivo, se utilizaron tasks de Verilog para las salidas para el usuario.